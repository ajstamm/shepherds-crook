% Options for packages loaded elsewhere
\PassOptionsToPackage{unicode}{hyperref}
\PassOptionsToPackage{hyphens}{url}
%
\documentclass[
  a5paper,
]{article}
\usepackage{amsmath,amssymb}
\usepackage{lmodern}
\usepackage{iftex}
\ifPDFTeX
  \usepackage[T1]{fontenc}
  \usepackage[utf8]{inputenc}
  \usepackage{textcomp} % provide euro and other symbols
\else % if luatex or xetex
  \usepackage{unicode-math}
  \defaultfontfeatures{Scale=MatchLowercase}
  \defaultfontfeatures[\rmfamily]{Ligatures=TeX,Scale=1}
  \setmainfont[]{Arial}
\fi
% Use upquote if available, for straight quotes in verbatim environments
\IfFileExists{upquote.sty}{\usepackage{upquote}}{}
\IfFileExists{microtype.sty}{% use microtype if available
  \usepackage[]{microtype}
  \UseMicrotypeSet[protrusion]{basicmath} % disable protrusion for tt fonts
}{}
\makeatletter
\@ifundefined{KOMAClassName}{% if non-KOMA class
  \IfFileExists{parskip.sty}{%
    \usepackage{parskip}
  }{% else
    \setlength{\parindent}{0pt}
    \setlength{\parskip}{6pt plus 2pt minus 1pt}}
}{% if KOMA class
  \KOMAoptions{parskip=half}}
\makeatother
\usepackage{xcolor}
\IfFileExists{xurl.sty}{\usepackage{xurl}}{} % add URL line breaks if available
\IfFileExists{bookmark.sty}{\usepackage{bookmark}}{\usepackage{hyperref}}
\hypersetup{
  pdftitle={Shepherd's crook},
  pdfauthor={Abigail Stamm},
  hidelinks,
  pdfcreator={LaTeX via pandoc}}
\urlstyle{same} % disable monospaced font for URLs
\usepackage[margin=1in]{geometry}
\usepackage{longtable,booktabs,array}
\usepackage{calc} % for calculating minipage widths
% Correct order of tables after \paragraph or \subparagraph
\usepackage{etoolbox}
\makeatletter
\patchcmd\longtable{\par}{\if@noskipsec\mbox{}\fi\par}{}{}
\makeatother
% Allow footnotes in longtable head/foot
\IfFileExists{footnotehyper.sty}{\usepackage{footnotehyper}}{\usepackage{footnote}}
\makesavenoteenv{longtable}
\usepackage{graphicx}
\makeatletter
\def\maxwidth{\ifdim\Gin@nat@width>\linewidth\linewidth\else\Gin@nat@width\fi}
\def\maxheight{\ifdim\Gin@nat@height>\textheight\textheight\else\Gin@nat@height\fi}
\makeatother
% Scale images if necessary, so that they will not overflow the page
% margins by default, and it is still possible to overwrite the defaults
% using explicit options in \includegraphics[width, height, ...]{}
\setkeys{Gin}{width=\maxwidth,height=\maxheight,keepaspectratio}
% Set default figure placement to htbp
\makeatletter
\def\fps@figure{htbp}
\makeatother
\setlength{\emergencystretch}{3em} % prevent overfull lines
\providecommand{\tightlist}{%
  \setlength{\itemsep}{0pt}\setlength{\parskip}{0pt}}
\setcounter{secnumdepth}{5}
\usepackage{xcolor}
\usepackage{multicol}
\PassOptionsToPackage{dvipsnames}{color}
\usepackage{fancyhdr}
\pagestyle{fancy}
\usepackage[fontsize=11pt]{scrextend}
\usepackage{enumitem}
\setlist{noitemsep,topsep=1pt,parsep=1pt,partopsep=1pt}
\setlength{\abovedisplayskip}{0pt}
\setlength{\belowdisplayskip}{0pt}
\setlength{\abovedisplayshortskip}{0pt}
\setlength{\belowdisplayshortskip}{0pt}
\setlength{\topskip}{0pt}
\setlength{\topsep}{5pt}
\setlength{\baselineskip}{0pt}
\setlength{\itemsep}{5pt}
\setlength{\headsep}{5pt}
\setlength{\parsep}{0pt}
\setlength{\parskip}{11pt}
\setlength{\partopsep}{0pt}
\usepackage{booktabs}
\usepackage{longtable}
\usepackage{array}
\usepackage{multirow}
\usepackage{wrapfig}
\usepackage{float}
\usepackage{colortbl}
\usepackage{pdflscape}
\usepackage{tabu}
\usepackage{threeparttable}
\usepackage{threeparttablex}
\usepackage[normalem]{ulem}
\usepackage{makecell}
\usepackage{xcolor}
\ifLuaTeX
  \usepackage{selnolig}  % disable illegal ligatures
\fi

\title{Shepherd's crook}
\author{Abigail Stamm}
\date{}

\begin{document}
\maketitle

{
\setcounter{tocdepth}{1}
\tableofcontents
}
\pagebreak

\(~\)

\(~\)

A solo contemplative game about shepherding your flock.
This game is based on the Lost and Found system\\
(\url{https://lostandfound.games}) created by Jack Harrison.

\pagebreak

\(~\)

\hypertarget{preparing-the-game}{%
\section{Preparing the game}\label{preparing-the-game}}

\hypertarget{set-up}{%
\subsection{Set up}\label{set-up}}

Before you start, you need:

\begin{itemize}
\tightlist
\item
  A quiet, peaceful place and at least 15 minutes per chapter
\item
  Paper and implements to draw and write, such as pencils and paintbushes
\item
  A time keeper, such as a clock or timer
\end{itemize}

\(~\)

\hypertarget{safety}{%
\subsection{Safety}\label{safety}}

As a solo activity, this game does not require you to negotiate boundaries and acceptable topics with others. However, you may find it helpful to consider your own boundaries. If at any time, this game triggers feelings of discomfort or loneliness, stop and step away until you feel comfortable enough to continue.

\pagebreak

\(~\)

\hypertarget{key-concepts}{%
\section{Key concepts}\label{key-concepts}}

\begin{enumerate}
\def\labelenumi{\arabic{enumi}.}
\item
  \textbf{Staff}\\
  The object whose story you are telling. The staff cannot die, but may change as it passes from one keeper to another. You are the staff.
\item
  \textbf{Keeper}\\
  The staff's owner or wielder. The staff can only act through the wielder, not on its own. Each keeper represents a different chapter in the staff's story.
\item
  \textbf{Rest}\\
  You can choose whether the staff passes directly from one keeper to another or whether, sometimes, it is set aside and rests until the next keeper comes along. How long it rests and how its new weilder obtains it are up to you.
\item
  \textbf{Traits}\\
  Each new staff begins with three traits of your choosing, either from the list provided or your own creativity. These traits may represent magical, mundane, physical, or personality attributes. Over time, these traits will evolve and change.
\item
  \textbf{Change}\\
  Every time your staff passes to a new keeper, it changes in some way, perhaps due to the influence of the previous keeper or the manner in which it was lost. These changes may be to the staff itself, to the staff's reality or traits, or in response to periodic prompts. You can make additional changes at any time that are as large or small as you'd like.
\end{enumerate}

\pagebreak

\(~\)

\hypertarget{game-structure}{%
\subsection{Game structure}\label{game-structure}}

\emph{Beginning:}\\
Your staff's origin story, creator, and first keeper (if they are not your creator)

\emph{Chapters:}\\
Each chapter follows a separate wielder. I suggest writing one wielder per sitting or day in as much or as little detail as you would like, but the choice is yours.

\begin{itemize}
\item
  \emph{Keeper:}\\
  Describe who they are and how they acquired you.
\item
  \emph{Event:}\\
  Describe one notable event in which you and your keeper were involved.
\item
  \emph{Rest:}\\
  Describe how they were parted from you and note how long you waited for your next keeper.
\end{itemize}

\emph{Ending:}\\
Spend some time in quiet contemplation. Describe your feelings and traits after your final keeper.

\pagebreak

\(~\)

\hypertarget{your-identity}{%
\section{Your identity}\label{your-identity}}

You may choose from three types of staff. While each is described separately, feel free to mix and match to fit your preferred style or the story you want to tell. From this point forward, ``you'' refers to the staff whose story you are telling.

\begin{enumerate}
\def\labelenumi{\arabic{enumi}.}
\item
  \textbf{Staff of office}\\
  You are a symbolic staff for rulers, prophets, and religious leaders.
\item
  \textbf{Staff of power}\\
  You are a magical staff for wizards, hedge witches, and others who practice magic.
\item
  \textbf{Staff of shepherding}\\
  You are a shepherd's crook used largely by herders, farmers, and others who tend livestock.
\end{enumerate}

Now choose three traits to describe yourself, either from Table \ref{tab:attr} or from your imagination. If you want to impose some randomness, roll some D20s (20-sided dice) and select three options from the corresponding rows in the table below.

After selecting your traits, draw your appearance. Feel free to be as silly or serious, simple or ornate, cartoonish or realistic as you would like. You need not draw well or spend much time on this step. Feel free to add colors and descriptive words as well. Consider whether you have been embellished with carvings, paint, jewels, beads, metal, or other adornments.

\pagebreak

\(~\)

\begin{table}

\caption{\label{tab:attr}Staff attributes}
\centering
\fontsize{9}{11}\selectfont
\begin{tabular}[t]{rllll}
\toprule
\textbf{roll} & \textbf{office} & \textbf{power} & \textbf{shepherd} & \textbf{general}\\
\midrule
\cellcolor{gray!6}{1} & \cellcolor{gray!6}{alluring} & \cellcolor{gray!6}{aggressive} & \cellcolor{gray!6}{adaptable} & \cellcolor{gray!6}{liberating}\\
2 & beneficent & blessed & clever & conservative\\
\cellcolor{gray!6}{3} & \cellcolor{gray!6}{cruel} & \cellcolor{gray!6}{capricious} & \cellcolor{gray!6}{compassionate} & \cellcolor{gray!6}{radical}\\
4 & delicate & controlling & cryptic & content\\
\cellcolor{gray!6}{5} & \cellcolor{gray!6}{discerning} & \cellcolor{gray!6}{curious} & \cellcolor{gray!6}{destined} & \cellcolor{gray!6}{serene}\\
\addlinespace
6 & enlightened & cursed & devious & ecclectic\\
\cellcolor{gray!6}{7} & \cellcolor{gray!6}{folksy} & \cellcolor{gray!6}{divine} & \cellcolor{gray!6}{doomed} & \cellcolor{gray!6}{quirky}\\
8 & gentle & driven & eccentric & silly\\
\cellcolor{gray!6}{9} & \cellcolor{gray!6}{healing} & \cellcolor{gray!6}{eager} & \cellcolor{gray!6}{haunted} & \cellcolor{gray!6}{ridiculous}\\
10 & humble & elegant & impatient & guilty\\
\addlinespace
\cellcolor{gray!6}{11} & \cellcolor{gray!6}{inspiring} & \cellcolor{gray!6}{horrific} & \cellcolor{gray!6}{isolating} & \cellcolor{gray!6}{beguiling}\\
12 & judging & insane & manipulative & mischevious\\
\cellcolor{gray!6}{13} & \cellcolor{gray!6}{loyal} & \cellcolor{gray!6}{ornery} & \cellcolor{gray!6}{obedient} & \cellcolor{gray!6}{beloved}\\
14 & mesmerizing & protective & patient & blighted\\
\cellcolor{gray!6}{15} & \cellcolor{gray!6}{opulent} & \cellcolor{gray!6}{resilient} & \cellcolor{gray!6}{practical} & \cellcolor{gray!6}{draining}\\
\addlinespace
16 & pastoral & stalwart & pragmatic & vampiric\\
\cellcolor{gray!6}{17} & \cellcolor{gray!6}{resourceful} & \cellcolor{gray!6}{tender} & \cellcolor{gray!6}{protective} & \cellcolor{gray!6}{parasitic}\\
18 & stoic & unassuming & romantic & corrosive\\
\cellcolor{gray!6}{19} & \cellcolor{gray!6}{terrifying} & \cellcolor{gray!6}{vengeful} & \cellcolor{gray!6}{sturdy} & \cellcolor{gray!6}{wary}\\
20 & unadorned & wise & unfeeling & unfulfilled\\
\bottomrule
\end{tabular}
\end{table}

\pagebreak

\(~\)

\hypertarget{your-origins}{%
\section{Your origins}\label{your-origins}}

After you have described your appearance and traits, consider your origins.

\begin{enumerate}
\def\labelenumi{\arabic{enumi}.}
\tightlist
\item
  Where were you made, carved, or decorated? A forge, forest, farm, or somewhere else?
\item
  Who created you? Are they well known for their craft? Were they commissioned?
\item
  Why were you created? What was your creator's motivation? What was your intended purpose?
\end{enumerate}

Now that you know a bit about where you came from, consider some other major elements of your story. You may not know the answers to these questions yet, so feel free to jot down a few notes and return to them at any time.

\begin{enumerate}
\def\labelenumi{\arabic{enumi}.}
\tightlist
\item
  What are you called? Do you have a formal name or title?
\item
  What is your relationship with your keepers? Master/slave, mentor/student, hero/sidekick, worker/tool, familial, antagonistic, synergistic?
\item
  What place would you most like to return to? It this place physical, temporal, psychological? Conversely, what place do you most want to avoid?
\item
  Describe an event in which you and a keeper were involved that later became a ballad, folktale, or parable. How you you feel about the story's evolution over time? What did it get right? What did it get wrong?
\item
  You and a keeper were involved in a love story. What role did your keeper play - lover, beloved, rival, officiant, obstacle? What was your part in the story?
\item
  As an immortal object, what do you find most baffling about your keepers' behaviors, values, and fears? If you could speak, what would you want to share with them, if anything?
\end{enumerate}

After you have considered these questions, close your eyes and see yourself in your mind. For about a minute, or longer if you'd like, silently contemplate how your beginnings relate to the traits you selected. When you have finished contemplating, write down thoughts you would like to revisit and prepare to meet your first keeper.

\pagebreak

\(~\)

\hypertarget{your-first-keeper}{%
\subsection{Your first keeper}\label{your-first-keeper}}

Choose your first keeper from the following list. If you are not sure what to choose, roll a D6 (6-sided die) and select the resulting number. Each keeper type includes a range of suggestions. At least one of the suggestions for each keeper type should be suitable for the type of staff you chose to be.

\begin{enumerate}
\def\labelenumi{\arabic{enumi}.}
\item
  \textbf{Peasant}\\
  Your keeper came from humble beginnings. Perhaps they were a lowly shepherd, an acolyte at the village temple, or a servant or slave in a noble's manor. Were you a tool of their labor, a key to their salvation, or both?
\item
  \textbf{Noble}\\
  Your keeper was high-born. Perhaps they were a noble, royal, or well-placed military officer. Were you a badge of their rank or a tool of their punishment?
\item
  \textbf{Relgious}\\
  Your keeper was a member of a religious order. Perhaps they were a lowly monk or nun toiling in the fields, or someone more highly placed, like an abbot or priestess. What purpose did you serve for them?
\item
  \textbf{Rogue}\\
  Your keeper lived outside the law. Perhaps they were an outlaw or thief outwitting their would-be captors, or a well-placed business owner or civil servant for whom the law did not apply. How did you help them evade capture?
\item
  \textbf{Hermit}\\
  Your keeper enjoyed solitude. Perhaps they lived on the edge of society, or in the wilderness, tending their flocks. Were they isolated by choice or did their profession or reputation require it?
\item
  \textbf{Healer}\\
  Your keeper served others. Perhaps they were a midwife, doctor, hedge witch, or apprentice. How were they regarded by their community?
\end{enumerate}

\pagebreak

\(~\)

\hypertarget{events}{%
\section{Events}\label{events}}

After describing each new keeper, select an event from the list provided. When you have completed one event, if you would like to explore your keeper further, feel free to return to the list to select a second or third event. After you resolve each event, change your identity in some small way. Add or remove or change a trait, but make sure you always have three to seven traits, or change your appearance somehow.

\emph{Suggestion: Each event for any given keeper should be different. Do not select more than three events for any one keeper and do not select the same event type twice in a row. For any six events, try to include each event type at least once.}

\textbf{Visitor}\\
The visitor may be welcome, expected, or undesired. How does your keeper's meeting with them go? How do they interact with the community and how are you involved? Choose from the list below or roll a D6 and select an element from the corresponding number.

\begin{enumerate}
\def\labelenumi{\arabic{enumi}.}
\tightlist
\item
  Vendor: an established trader or passing journeyman bringing news and goods
\item
  Entertainer: visiting performers, fortune tellers, or charlitans
\item
  Educator: a new teacher, scientist, or researcher
\item
  Tourist: a traveler, authority, or friend from outside the community, or a wandering lost sheep
\item
  Dissident: someone who threatens established authority
\item
  Seeker: an adventurer, romantic, or other lost soul
\end{enumerate}

\textbf{Crisis}\\
The crisis may be an internal or external threat. How does your keeper use you to resolve it? How are you affected by it? Choose from the list below or roll a D6 and select an element from the corresponding number.

\begin{enumerate}
\def\labelenumi{\arabic{enumi}.}
\tightlist
\item
  Disaster: something natural, such as a flood or earthquake
\item
  Hunger: a food shortage from drought, flooding, spoiling, or mismanagement
\item
  Violence: war, a raiding party, or a predator
\item
  Unseenn: something invisible, like a pathogen or a dangerous idea
\item
  Insidious: a new craze or a fast-spreading plant or animal species
\item
  Threat: challenge to your keeper's authority or necessity, such as a new shepherd, a rival for their position, or their protege surpassing them
\end{enumerate}

\textbf{Pastoral care}\\
Pastoral care is important for any shepherd and can take many forms. What are your and your keeper's roles in different aspects of pastoral care? How do you both feel about it? Are there specific traditions associated with them? Choose from the list below or roll a D6 and select an element from the corresponding number.

\begin{enumerate}
\def\labelenumi{\arabic{enumi}.}
\tightlist
\item
  Births: including twins, stillbirths, infants with deformities
\item
  Deaths: from injury, disease, or childbirth; mercy killing
\item
  Illness: including cures, palliative care, superstition, and quarantine
\item
  Injury: treatment and what to do when treatment fails
\item
  Mediation: including negotiation, conflict, domestic disputes
\item
  Mental health: of yourself, your keeper, or members of the community
\end{enumerate}

\textbf{Observation}\\
Sometimes, nothing of particular interest happens. Take some time to describe one aspect of your home or community. What are your and your keeper's rolesin the community? How do you both feel about it? Choose from the list below or roll a D6 and select an element from the corresponding number.

\begin{enumerate}
\def\labelenumi{\arabic{enumi}.}
\tightlist
\item
  Environment: habitat, community size, technological advancement
\item
  Home: where and with whom your keeper lives
\item
  Duties: shepherding, leading, protecting
\item
  Politics: type of society, how leaders are selected and laws are made and enforced
\item
  Foods: the most and least popular, how agriculture and markets work
\item
  Arts: clothes, hobbies, entertainment
\end{enumerate}

\pagebreak

\(~\)

\hypertarget{rest}{%
\section{Rest}\label{rest}}

After your final event with your keeper, describe how you were parted from them. If you need ideas, choose from the list below or roll a D6 and select an element from the corresponding number.

\begin{enumerate}
\def\labelenumi{\arabic{enumi}.}
\tightlist
\item
  Your keeper died and left you to a relative or protege. Did you end up with the intended recipient?
\item
  Your keeper passed you on to someone else. Why did they give you away? Who did they give you to?
\item
  Your keeper sold you or auctioned you off. Did your keeper want to sell you? How did you feel about it?
\item
  You were stolen. Was the theft violent? Were you stolen by someone your keeper trusted?
\item
  You were lost. Were you traveling? Was your keeper careless? How were you found?
\item
  You were catalogued and placed in storage or on display. Why were you set aside? How were you returned to service?
\end{enumerate}

Change your identity in a major way (appearance, purpose, trait, or name, or some combination thereof) or change an important aspect of your world. Decide how long you will be dormant between keepers. Minutes? Weeks? Years? Centuries?

Close your eyes and sit in silence for one to five minutes depending on the length of your dormancy. Take a break and do something else, or wait until the next day, before you return to the keeper list to select a new keeper.

\pagebreak

\(~\)

\hypertarget{an-ending-of-sorts}{%
\section{An ending of sorts}\label{an-ending-of-sorts}}

This is the end of your story. When you have parted ways with your last keeper, close your eyes and sit in silence for five minutes contemplating your life. Then write some final thoughts on how you will be remembered. If you have not already, return to the list of questions in \protect\hyperlink{your-origins}{Your origins} and try to answer any remaining. Perhaps you will live on as an item or a legend in a future role playing game or story.

\end{document}
